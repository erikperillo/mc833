\documentclass[11pt]{article}
\usepackage{graphicx}
\usepackage{float}
\usepackage{amsmath}
\usepackage{amsfonts}
\usepackage[brazilian]{babel}
\usepackage[utf8]{inputenc}
\usepackage[T1]{fontenc}

%macros
\newcommand{\fromeng}[1]{\footnote{do inglês: \textit{#1}}}
\newcommand{\tit}[1]{\textit{#1}}
\newcommand{\tbf}[1]{\textbf{#1}}
\newcommand{\ttt}[1]{\texttt{#1}}

\begin{document}

\title{MC833 - Exercício 1}
\author{Erik de Godoy Perillo - RA: 135582}
\maketitle

\begin{enumerate}
\item --
\begin{enumerate}
	\item O parâmetro \ttt{-c} especifica quantas vezes o programa irá alcançar 
		o host especificado para contar os tempos de ida e volta.
		Os tempos mínimo, médio e máximo para \ttt{www.cam.ac.uk} foram, 
		respectivamente, \ttt{204.025, 214.034, 224.255} ms.
	\item Os tempos para \ttt{www.unicamp.br}, como esperado, foram muito
		menores! Os tempos mínimo, médio e máximo foram, respectivamente,
		\ttt{0.873, 6.438, 17.095} ms. 
	\item Não, pois foi possível acessar o site pelo \tit{browser} mas não 
		foi possível alcançá-lo pelo \ttt{ping}.
\end{enumerate}

\item Há a interface \ttt{wlp3s0}.
	Seus endereços IP \ttt{inet, netmask, broadcast} são, respectivamente,
	\ttt{177.220.89.54, 255.255.254.0, 177.220.89.255}.
	Foram recebidos $2080396093$ \tit{bytes} e enviados $130064906$ 
	\tit{bytes}. Há também a interface \ttt{lo}, com endereço IP
	\ttt{inet 127.0.0.1}. Foram enviados e recebidos $7421153$ \tit{bytes}.

\item Foram enviados e recebidos $13245$ pacotes. Após o comando 
	\ttt{ping -c 2 127.0.0.1}, a quantidade de pacotes subiu para $13267$.
	Assim, houve um acréscimo de $22$ pacotes. Não posso tirar conclusões
	sobre a correlação entre pacotes enviados pelo comando \ttt{ping}
	e o valor de pacotes que aparecem a mais com \ttt{ifconfig lo} pois
	notei que o número de pacotes se altera mesmo quando não uso ping, ou
	seja, pode haver outros fatores que influenciam nesse número.

\item Há três rotas. A padrão (\ttt{default}) tem \tit{gateway} 
	\ttt{wifig-gw.ifi.un} na interface \ttt{slp3s0}.

\item --
\begin{enumerate}
	\item O endereço encontrado para \ttt{www.google.com} foi 
		\ttt{172.217.28.36}. Mais de um endereço IP indica múltiplos servidores
		autoritativos que podem lidar com mais dados. O servidor DNS da minha
		estação tem o IP \ttt{143.106.9.2}.
	\item Este endereço diz respeito ao meu próprio computador e tem o nome
		\ttt{localhost}.
\end{enumerate}

\item --
\begin{enumerate}
	\item Há nove roteadores. Pelos nomes (terminador em .br), há seis deles
		no Brasil.
	\item Há 17 roteadores neste caso. Os quatro primeiros são comuns aos dois
		casos.
	\item Com um limite máximo de 255 hops, ele usou todos (uau). 
		Não, não é a mesma, é feita apenas por 17 roteadores.
	\item A partir do oitavo hop, \ttt{redclara.lon.uk.geant.net}. Isso porque
		o \tit{rtt} desse hop ficou entre 200ms, enquanto que os anteriores
		a ele não passaram de 20ms.
\end{enumerate}

\item --
\begin{enumerate}
	\item São dadas informações em tempo real sobre as conexões TCP feitas
		enquanto acessa-se o site da Unicamp.
	\item Não vi nenhuma.
	\item O campo \tit{Foreign Address} muitas vezes começa com 
		\ttt{gru06}.
\end{enumerate}

\item --
\begin{enumerate}
	\item Simplesmente especifico a porta padrão HTTP:
		\ttt{telnet www.google.com 80}
	\item Não é possível, é necessário um serviço nessa porta para lidar com a 
		comunicação.
\end{enumerate}

\end{enumerate}

\end{document}
