\documentclass[11pt]{article}
\usepackage{graphicx}
\usepackage{float}
\usepackage{amsmath}
\usepackage{amsfonts}
\usepackage[brazilian]{babel}
\usepackage[utf8]{inputenc}
\usepackage[T1]{fontenc}

%macros
\newcommand{\fromeng}[1]{\footnote{do inglês: \textit{#1}}}
\newcommand{\tit}[1]{\textit{#1}}
\newcommand{\tbf}[1]{\textbf{#1}}
\newcommand{\ttt}[1]{\texttt{#1}}

\begin{document}

\title{MC833 - Tarefa 2}
\author{Erik de Godoy Perillo - RA: 135582}
\maketitle

\begin{enumerate}

\item As interfaces pelas quais pode-se capturar dados podem ser listadas
com a \tit{flag} \ttt{--list-interfaces}:

\fbox{\parbox{\textwidth}{
\ttt{\$ tcpdump ---list-interfaces\newline
1.wlp3s0 [Up, Running]\newline
2.any (Pseudo-device that captures on all interfaces) [Up, Running]\newline
3.lo [Up, Running, Loopback]\newline
4.enp2s0 [Up]\newline
5.bluetooth-monitor (Bluetooth Linux Monitor)\newline
6.nflog (Linux netfilter log (NFLOG) interface)\newline
7.nfqueue (Linux netfilter queue (NFQUEUE) interface)\newline
8.dbus-system (D-Bus system bus)\newline
9.dbus-session (D-Bus session bus)\newline
10.usbmon1 (USB bus number 1)\newline
11.usbmon2 (USB bus number 2)\newline
12.usbmon3 (USB bus number 3)\newline
}}}

São muito mais interfaces que as mostradas como disponíveis pelo \ttt{ifconfig}:

\fbox{\parbox{\textwidth}{
\ttt{\$ ifconfig -a -s\newline
Iface      MTU    RX-OK RX-ERR RX-DRP RX-OVR    
TX-OK TX-ERR TX-DRP TX-OVR Flg\newline
enp2s0    1500        0      0      
0 0             0      0      0      0 BMU\newline
lo       65536     4433      0      0 0          
4433      0      0      0 LRU\newline
wlp3s0    1500   842849      0      0 0        
325951      0      0      0 BMRU\newline
}}}

\item Usando-se a \tit{flag} \ttt{-nn}, consegue-se ver os endereços IP
ao invés dos nomes dos domínios dos \tit{hosts}.

\fbox{\parbox{\textwidth}{
\ttt{\$ tcpdump -r tcpdump.dat | tail -n1\newline
ding from file tcpdump.dat, link-type EN10MB (Ethernet)\newline
01:34:44.339015 IP willow.csail.mit.edu.39675 > 
maple.csail.mit.edu.commplex-link: Flags [.], ack 2, win 115, options 
[nop,nop,TS val 282139339 ecr 282204955], length 0
}}}
\fbox{\parbox{\textwidth}{
\ttt{\$ tcpdump -nnr tcpdump.dat | tail -n1\newline
ding from file tcpdump.dat, link-type EN10MB (Ethernet)\newline
01:34:44.339015 IP 128.30.4.222.39675 > 128.30.4.223.5001: 
Flags [.], ack 2, win 115, options [nop,nop,TS val 282139339 
ecr 282204955], length 0
}}}

Assim, os endereços IP de \tit{willow} e \tit{maple} são, respectivamente,
\ttt{128.30.4.222} e \ttt{128.30.4.223}.

\item --

\fbox{\parbox{\textwidth}{
\ttt{\$ tcpdump -e -r tcpdump.dat | tail -n1\newline
ding from file tcpdump.dat, link-type EN10MB (Ethernet)\newline
01:34:44.339015 00:16:ea:8e:28:44 (oui Unknown) > 00:16:ea:8d:e5:8a 
(oui Unknown), ethertype IPv4 (0x0800), length 66: willow.csail.mit.edu.39675 > 
maple.csail.mit.edu.commplex-link: Flags [.], ack 2, win 115, options 
[nop,nop,TS val 282139339 ecr 282204955], length 0
}}}

Assim, os endereços MAC de \tit{willow} e \tit{maple} são, respectivamente,
\ttt{00:16:ea:8e:28:44} e \ttt{00:16:ea:8d:e5:8a}.

\item --

\fbox{\parbox{\textwidth}{
\ttt{\$ tcpdump -nnr tcpdump.dat | tail -n1\newline
ding from file tcpdump.dat, link-type EN10MB (Ethernet)\newline
01:34:44.339015 IP 128.30.4.222.39675 > 128.30.4.223.5001: Flags [.], ack 2, 
win 115, options [nop,nop,TS val 282139339 ecr 282204955], length 0
}}}

Assim, as portas de \tit{willow} e \tit{maple} são, respectivamente,
\ttt{39675} e \ttt{5001}.

\item --

\fbox{\parbox{\textwidth}{
\ttt{\$ tcpdump -r tcpdump.dat | tail -n2\newline
reading from file tcpdump.dat, link-type EN10MB (Ethernet)\newline
01:34:44.339007 IP maple.csail.mit.edu.commplex-link > 
willow.csail.mit.edu.39675: Flags [F.], seq 1, ack 1572890, win 905, options 
[nop,nop,TS val 282204955 ecr 282139320], length 0 01:34:44.339015 IP 
willow.csail.mit.edu.39675 > maple.csail.mit.edu.commplex-link: Flags [.], ack 
2, win 115, options [nop,nop,TS val 282139339 ecr 282204955], length 0
}}}

\fbox{\parbox{\textwidth}{
\ttt{\$ tcpdump -r tcpdump.dat | head -n1\newline
reading from file tcpdump.dat, link-type EN10MB (Ethernet)\newline
01:34:41.473036 ARP, Request who-has maple.csail.mit.edu tell 
willow.csail.mit.edu, length 28
}}}

\fbox{\parbox{\textwidth}{
\ttt{\$ tcpdump -r tcpdump.dat | tail -n1\newline
reading from file tcpdump.dat, link-type EN10MB (Ethernet)\newline
01:34:44.339015 IP willow.csail.mit.edu.39675 > 
maple.csail.mit.edu.commplex-link: Flags [.], ack 2, win 115, options 
[nop,nop,TS val 282139339 ecr 282204955], length 0
}}}

O último ACK é de número 1572890, assim foram transferidos 1572889 bytes.
A conexão começou em 1:34:41.47 e terminou em 1:34:44.34, assim ela durou
2.87 segundos. Deste modo, a vazão foi de $1572.889/2.87 = 548$ 
Kilobytes por segundo.

\item --

\fbox{\parbox{\textwidth}{
\ttt{\$ cat outfile.txt | grep "1473:2921|ack 2921"\newline
01:34:41.474225 IP willow.csail.mit.edu.39675 > 
maple.csail.mit.edu.commplex-link: Flags [.], seq 1473:2921, ack 1, win 115, 
options [nop,nop,TS val 282136474 ecr 282202089], length 1448\newline
01:34:41.482047 IP maple.csail.mit.edu.commplex-link > 
willow.csail.mit.edu.39675: Flags [.], ack 2921, win 159, options 
[nop,nop,TS val 282202095 ecr 282136474], length 0
}}}

Entre o envio e o ACK do recebimento, o tempo foi de $41.482 - 41.474 = 0.08$ 
segundos.

\fbox{\parbox{\textwidth}{
\ttt{\$ cat outfile.txt | grep "13057:14505|ack 14505"\newline
01:34:41.474992 IP willow.csail.mit.edu.39675 > 
maple.csail.mit.edu.commplex-link: Flags [.], seq 13057:14505, ack 1, 
win 115, options [nop,nop,TS val 282136474 ecr 282202090], length 1448\newline
01:34:41.499373 IP maple.csail.mit.edu.commplex-link > 
willow.csail.mit.edu.39675: Flags [.], ack 14505, win 331, options 
[nop,nop,TS val 282202114 ecr 282136474], length 0
}}}

O tempo foi de $41.499 - 41.475 = 0.24$ segundos. 
Diversos fatores podem ter causado a diferença de tempo. 

\item --

\tit{Three-way handshake}:

\fbox{\parbox{\textwidth}{
\ttt{
01:34:41.473518 IP willow.csail.mit.edu.39675 > 
maple.csail.mit.edu.commplex-link: Flags [S], seq 1258159963, win 14600, options 
[mss 1460,sackOK,TS val 282136473 ecr 0,nop,wscale 7], length 0\newline
01:34:41.474055 IP maple.csail.mit.edu.commplex-link > 
willow.csail.mit.edu.39675: Flags [S.], seq 2924083256, ack 1258159964, win 
14480, options [mss 1460,sackOK,TS val 282202089 ecr 282136473,nop,wscale 7], 
length 0\newline
01:34:41.474079 IP willow.csail.mit.edu.39675 > 
maple.csail.mit.edu.commplex-link: Flags [.], ack 1, win 115, options 
[nop,nop,TS val 282136474 ecr 282202089], length 0
}}}

\centering{
\begin{tabular}{| c | c | c | c |}
	\hline
	número & fonte & destino & mensagem\\
	\hline
	1 & willow & maple & SYN, seq=1258159963\\ 
	2 & maple & willow & SYN, ACK=1258159964, seq=2924083256\\ 
	3 & willow & maple & ACK=1\\ 
	\hline
\end{tabular}}

\tit{Connection termination}:

\fbox{\parbox{\textwidth}{
\ttt{
01:34:44.311921 IP willow.csail.mit.edu.39675 > 
maple.csail.mit.edu.commplex-link: Flags [FP.], seq 1572017:1572889, ack 1, 
win 115, options [nop,nop,TS val 282139311 ecr 282204927], length 872\newline
01:34:44.329956 IP maple.csail.mit.edu.commplex-link > 
willow.csail.mit.edu.39675: Flags [.], ack 1572890, win 820, options 
[nop,nop,TS val 282204945 ecr 282139320], length 0\newline
01:34:44.339007 IP maple.csail.mit.edu.commplex-link > 
willow.csail.mit.edu.39675: Flags [F.], seq 1, ack 1572890, win 905, options 
[nop,nop,TS val 282204955 ecr 282139320], length 0\newline
01:34:44.339015 IP willow.csail.mit.edu.39675 > 
maple.csail.mit.edu.commplex-link: Flags [.], ack 2, win 115, options 
[nop,nop,TS val 282139339 ecr 282204955], length 0
}}}

\centering{
\begin{tabular}{| c | c | c | c |}
	\hline
	número & fonte & destino & mensagem\\
	\hline
	1 & maple & willow & FYN\\
	2 & willow & maple & ACK\\
	3 & willow & maple & FYN\\
	4 & maple & maple & ACK\\
	\hline
\end{tabular}}

\end{enumerate}
\end{document}
