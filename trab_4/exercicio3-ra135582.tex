\documentclass[11pt]{article}
\usepackage{graphicx}
\usepackage{float}
\usepackage{amsmath}
\usepackage{amsfonts}
\usepackage[brazilian]{babel}
\usepackage[utf8]{inputenc}
\usepackage[T1]{fontenc}

%macros
\newcommand{\fromeng}[1]{\footnote{do inglês: \textit{#1}}}
\newcommand{\tit}[1]{\textit{#1}}
\newcommand{\tbf}[1]{\textbf{#1}}
\newcommand{\ttt}[1]{\texttt{#1}}

\begin{document}

\title{MC833 - Tarefa 3}
\author{Erik de Godoy Perillo - RA: 135582}
\maketitle

\begin{enumerate}

\item A função \ttt{connect} estabelece uma conexão entre um \tit{socket}
especificado e um host, cujas características (como família de endereços 
a ser usada, endereço em si, porta) são especificadas também (na implementação
em C para Linux, por exemplo é usado o apontador para uma \ttt{struct sockaddr}.
O tipo de conexão (protocolo a ser usado) varia de acordo com o socket 
especificado.

\item -- 

O teste foi executado primeiro rodando o servidor e depois o cliente.

Lado cliente (somente há entrada):

\fbox{\parbox{\textwidth}{
\ttt{\$ ./client localhost\newline
ey b0ss\newline
ey b0ss?\newline
 follow the white rabbit.
}}}

Lado servidor (somente há saída):

\fbox{\parbox{\textwidth}{
\ttt{\$ ./server\newline
ey b0ss\newline
ey b0ss?\newline
 follow the white rabbit.
}}}

\item 

Usando-se \ttt{netstat} enquanto está havendo a comunicação, pode-se comprovar
(observe o nome do processo referente à porta que estamos usando, 31472):

\fbox{\parbox{\textwidth}{
\ttt{\$ netstat -ap\newline
Active Internet connections (servers and established)\newline
Proto Recv-Q Send-Q Local Address           Foreign Address
         State       PID/Program name\newline
tcp        0      0 *:31472                 *:*                     LISTEN
      11423/./server\newline
tcp        0      0 *:ssh                   *:*                     
LISTEN\newline  
tcp        0      0 localhost.localdom:6342 *:*                     LISTEN
      623/megasync\newline
}}}

Usando-se \ttt{tcpdump}, também podemos checar. O \ttt{tcpdump} foi iniciado
com a conexão já iniciada e, após uma mensagem ser enviada pelo cliente,
houve saída pelo comando.

\fbox{\parbox{\textwidth}{
\ttt{\$ sudo tcpdump -i any -vvv "port 31472"\newline
tcpdump: listening on any, link-type LINUX\_SLL (Linux cooked), capture size 
262144 bytes\newline
12:23:35.652932 IP (tos 0x0, ttl 64, id 51944, offset 0, flags [DF], 
proto TCP (6), length 60)\newline
    localhost.localdomain.40918 > localhost.localdomain.31472: Flags [S], cksum 
0xfe30 (incorrect -> 0xc0ee), seq 3165645406, win 43690, options 
[mss 65495,sackOK,TS val 28647606 ecr 0,nop,wscale 7], length 0\newline
12:23:35.652974 IP (tos 0x0, ttl 64, id 0, offset 0, flags [DF], proto TCP (6), 
length 60)\newline
    localhost.localdomain.31472 > localhost.localdomain.40918: Flags [S.], 
cksum 0xfe30 (incorrect -> 0x5343), seq 1737876377, ack 3165645407, win 43690, 
options [mss 65495,sackOK,TS val 28647606 ecr 
28647606,nop,wscale 7], length 0\newline
12:23:35.653010 IP (tos 0x0, ttl 64, id 51945, offset 0, flags [DF], proto TCP 
(6), length 52)\newline
    localhost.localdomain.40918 > localhost.localdomain.31472: 
Flags [.], cksum 0xfe28 (incorrect -> 0x2588), seq 1, ack 1, 
win 342, options [nop,nop,TS val 28647606 ecr 28647606], length 0
}}}

\item Sim, é possível. Isso porque, no protocolo telnet, o cliente apenas
pega a entrada do usuário e a manda para o servidor, por meio de uma conexão 
TCP. Assim, como nosso servidor está preparado para lidar com as mensagens
enviadas, tudo funciona. Assim, basta especificar a porta.

Lado do cliente (telnet):

\fbox{\parbox{\textwidth}{
\ttt{\$ telnet localhost 31472\newline
Trying 127.0.0.1...\newline
Connected to localhost.\newline
Escape character is '\^{]}'. \newline
oie\newline
 fascinante, legal mesmo, estou impressionado.
}}}

Lado do servidor:

\fbox{\parbox{\textwidth}{
\ttt{\$ ./server\newline
oie\newline
 fascinante, legal mesmo, estou impressionado.
}}}

\item Este foi resolvido no Susy. :\^). Um exemplo da brincadeira:

Lado servidor:

\fbox{\parbox{\textwidth}{
\ttt{\$ ./server\newline
oi\newline
testeee\newline
echoooooo\newline
uau!!!!!!!!!!!!!!!!!!
}}}

Lado cliente:

\fbox{\parbox{\textwidth}{
\ttt{\$ ./client\newline
oie\newline
oie\newline
testeee\newline
testeee\newline
echoooooo\newline
echoooooo\newline
uau!!!!!!!!!!!!!!!!!!\newline
uau!!!!!!!!!!!!!!!!!!
}}}

 
\end{enumerate}
\end{document}
