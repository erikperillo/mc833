\documentclass[11pt]{article}
\usepackage{graphicx}
\usepackage{float}
\usepackage{amsmath}
\usepackage{amsfonts}
\usepackage[brazilian]{babel}
\usepackage[utf8]{inputenc}
\usepackage[T1]{fontenc}

%macros
\newcommand{\fromeng}[1]{\footnote{do inglês: \textit{#1}}}
\newcommand{\tit}[1]{\textit{#1}}
\newcommand{\tbf}[1]{\textbf{#1}}
\newcommand{\ttt}[1]{\texttt{#1}}

\begin{document}

\title{MC833 - Tarefa 4}
\author{Erik de Godoy Perillo - RA: 135582}
\maketitle

\begin{enumerate}

\item Após modificações no código, a saída ficou deste modo:

\fbox{\parbox{\textwidth}{
\ttt{created socket on port \#64136, IP 127.0.0.1
}}}
 
\item Após modificações no código, a saída ficou deste modo:

\fbox{\parbox{\textwidth}{
\ttt{connected to IP 127.0.0.1, port \#64136
}}}

\item Uma simples chamada a \ttt{fork()} pode fazer o trabalho. 
	O processo inicial do servidor, então, serve como um despachante. 
	A cada nova conexão que ele estabelece, um novo processo é criado para ela.
	No código, checa-se se o pid retornado por fork é zero. 
	Neste caso, o processo foi despachado e então a tarefa da conexão é feita
	(nesse caso, receber strings do cliente e colocá-las na saída padrão).

\item Devemos lembrar que o valor retornado por \ttt{socket()} é um simples
	\tit{file descriptor}. 
	Assim, ele segue o comportamento do Unix:
	processos filhos herdam os \tit{file descriptors} do pai, o que são apenas
	referências aos arquivos em si. 
	Assim, fechar um socket que você não usa em um processo após um \ttt{fork()}
	não interfere no outro processo que o usa e é até necessário para que 
	não fiquem referências ao arquivo em aberto depois dos processos terminarem.

\item --
	\begin{enumerate}
		\item Foram usados três clientes.

		Cliente 1:

		\fbox{\parbox{\textwidth}{
		\ttt{\$ ./client localhost\newline
		created socket on port \#35966, IP 127.0.0.1\newline
		oi1
		}}}

		Cliente 2:

		\fbox{\parbox{\textwidth}{
		\ttt{\$ ./client localhost\newline
		created socket on port \#35968, IP 127.0.0.1\newline
		oii2
		}}}

		Cliente 3:

		\fbox{\parbox{\textwidth}{
		\ttt{\$ ./client localhost\newline
		created socket on port \#35970, IP 127.0.0.1\newline
		oiii3
		}}}

		Sendo as mensagens oi* entradas. 
		Do lado do servidor, após chamá-lo e invocar os clientes mandando
		mensagens como especificado acima, sua saída ficou assim:

		\fbox{\parbox{\textwidth}{
		\ttt{\$ ./server\newline
		[PID 4591] server listening on port \#31472\newline
		connected to IP 127.0.0.1, port \#35966\newline
		connected to IP 127.0.0.1, port \#35968\newline
		connected to IP 127.0.0.1, port \#35970\newline
		[PID 4593][IP 127.0.0.1, port 32396] oi1\newline
		[PID 4595][IP 127.0.0.1, port 32908] oii2\newline
		[PID 4597][IP 127.0.0.1, port 33420] oiii3
		}}}

		Ao mesmo tempo em que as mensagens eram enviadas (na ordem em que
		foram aqui mostradas), o comando \ttt{tcpdump -i lo} estava em ação:

		\fbox{\parbox{\textwidth}{
		\ttt{\$ sudo tcpdump -i lo\newline
		tcpdump: verbose output suppressed, use -v or -vv for full protocol 
		decode\newline
		listening on lo, link-type EN10MB (Ethernet), capture size 262144 bytes
		08:11:01.279942 IP localhost.localdomain.35966 > 
		localhost.localdomain.31472: Flags [P.], seq 3627854710:3627854715, ack 
		2242275928, win 342, options [nop,nop,TS val 18524259 ecr 18521305], 
		length 5\newline
		08:11:01.279982 IP localhost.localdomain.31472 > 
		localhost.localdomain.35966: Flags [.], ack 5, win 342, options 
		[nop,nop,TS val 18524259 ecr 18524259], length 0\newline
		08:11:03.577935 IP localhost.localdomain.35968 > 
		localhost.localdomain.31472: Flags [P.], seq 2490601672:2490601678, ack 
		3107615614, win 342, options [nop,nop,TS val 18524948 ecr 18521909], 
		length 6\newline
		08:11:03.577977 IP localhost.localdomain.31472 > 
		localhost.localdomain.35968: Flags [.], ack 6, win 342, options 
		[nop,nop,TS val 18524948 ecr 18524948], length 0\newline
		08:11:06.019491 IP localhost.localdomain.35970 > 
		localhost.localdomain.31472: Flags [P.], seq 3652956810:3652956817, ack 
		2238686682, win 342, options [nop,nop,TS val 18525681 ecr 18522355], 
		length 7\newline
		08:11:06.019582 IP localhost.localdomain.31472 > 
		localhost.localdomain.35970: Flags [.], ack 7, win 342, options 
		[nop,nop,TS val 18525681 ecr 18525681], length 0
		}}}

		Como se nota, os pacotes TCP foram de fato enviados dos clientes
		ao servidor. O pacote do tempo 08:11:01.279982 serve ao cliente 1,
		o do tempo 08:11:03.577977 ao cliente 2 e o do tempo 08:11:06.019582 
		ao cliente 3 (todos são acks).
		
		\item Foi modificado o programa \ttt{server.c} de tal modo que o PID
			do processo inicial e o PID dos filhos fosse mostrado. 
			No exemplo abaixo, três clientes se conectaram ao servidor.

			\fbox{\parbox{\textwidth}{\ttt{\$ ./server\newline
				[PID 24167] server listening on port \#31472\newline
				connected to IP 127.0.0.1, port \#22666\newline
				connected to IP 127.0.0.1, port \#23178\newline
				connected to IP 127.0.0.1, port \#23690\newline
				[PID 24173][IP 127.0.0.1, port 23690] oi\newline
				[PID 24171][IP 127.0.0.1, port 23178] e ai glr\newline
				[PID 24169][IP 127.0.0.1, port 22666] opaa}}}

			Então temos quatro PIDs. Para checar se os três últimos são realmente
			filhos do primeiro, podemos usar a ferramenta \ttt{ps} para isso.
			A \tit{flag} \ttt{--ppid} lista os filhos do PID especificado:

			\fbox{\parbox{\textwidth}{
			\ttt{\$ ps --ppid 24167 -o pid\newline
				24169\newline
				24171\newline
				24173
			}}}
	\end{enumerate}

	\item Após o final da conexão, o lado do cliente fica no estado 
		\ttt{TIME\_WAIT}. Isso condiz, pois implementamos um servidor 
		concorrente que está a espera de outras conexões. 
		A conexão TCP é encerrada, o cliente fica em \ttt{TIME\_WAIT}
		e o servidor fica em \ttt{LISTEN} como sempre.
		
		A ferramenta \ttt{netstat} pode ser usada para averiguar isso.
		O servidor foi executado (na porta 31472) e o cliente abriu uma 
		conexão (na porta 36240). Após terminar o cliente, 
		usou-se a ferramenta:

		\fbox{\parbox{\textwidth}{
		\ttt{\$ sudo netstat -ap | grep ":36240|:31472"\newline
		tcp 0 0 *:31472 *:* LISTEN 5614/./server\newline
		tcp 0 0 localhost.localdo:36240 localhost.localdo:31472 TIME\_WAIT -
		}}}

		E de fato: a conexão referente ao cliente (pela porta 36240) está
		em estado \ttt{TIME\_WAIT}.

\end{enumerate}
\end{document}
